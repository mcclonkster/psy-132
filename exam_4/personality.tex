% Options for packages loaded elsewhere
% Options for packages loaded elsewhere
\PassOptionsToPackage{unicode}{hyperref}
\PassOptionsToPackage{hyphens}{url}
\PassOptionsToPackage{dvipsnames,svgnames,x11names}{xcolor}
%
\documentclass[
  letterpaper,
  DIV=11,
  numbers=noendperiod]{scrartcl}
\usepackage{xcolor}
\usepackage{amsmath,amssymb}
\setcounter{secnumdepth}{-\maxdimen} % remove section numbering
\usepackage{iftex}
\ifPDFTeX
  \usepackage[T1]{fontenc}
  \usepackage[utf8]{inputenc}
  \usepackage{textcomp} % provide euro and other symbols
\else % if luatex or xetex
  \usepackage{unicode-math} % this also loads fontspec
  \defaultfontfeatures{Scale=MatchLowercase}
  \defaultfontfeatures[\rmfamily]{Ligatures=TeX,Scale=1}
\fi
\usepackage{lmodern}
\ifPDFTeX\else
  % xetex/luatex font selection
\fi
% Use upquote if available, for straight quotes in verbatim environments
\IfFileExists{upquote.sty}{\usepackage{upquote}}{}
\IfFileExists{microtype.sty}{% use microtype if available
  \usepackage[]{microtype}
  \UseMicrotypeSet[protrusion]{basicmath} % disable protrusion for tt fonts
}{}
\makeatletter
\@ifundefined{KOMAClassName}{% if non-KOMA class
  \IfFileExists{parskip.sty}{%
    \usepackage{parskip}
  }{% else
    \setlength{\parindent}{0pt}
    \setlength{\parskip}{6pt plus 2pt minus 1pt}}
}{% if KOMA class
  \KOMAoptions{parskip=half}}
\makeatother
% Make \paragraph and \subparagraph free-standing
\makeatletter
\ifx\paragraph\undefined\else
  \let\oldparagraph\paragraph
  \renewcommand{\paragraph}{
    \@ifstar
      \xxxParagraphStar
      \xxxParagraphNoStar
  }
  \newcommand{\xxxParagraphStar}[1]{\oldparagraph*{#1}\mbox{}}
  \newcommand{\xxxParagraphNoStar}[1]{\oldparagraph{#1}\mbox{}}
\fi
\ifx\subparagraph\undefined\else
  \let\oldsubparagraph\subparagraph
  \renewcommand{\subparagraph}{
    \@ifstar
      \xxxSubParagraphStar
      \xxxSubParagraphNoStar
  }
  \newcommand{\xxxSubParagraphStar}[1]{\oldsubparagraph*{#1}\mbox{}}
  \newcommand{\xxxSubParagraphNoStar}[1]{\oldsubparagraph{#1}\mbox{}}
\fi
\makeatother


\usepackage{longtable,booktabs,array}
\usepackage{calc} % for calculating minipage widths
% Correct order of tables after \paragraph or \subparagraph
\usepackage{etoolbox}
\makeatletter
\patchcmd\longtable{\par}{\if@noskipsec\mbox{}\fi\par}{}{}
\makeatother
% Allow footnotes in longtable head/foot
\IfFileExists{footnotehyper.sty}{\usepackage{footnotehyper}}{\usepackage{footnote}}
\makesavenoteenv{longtable}
\usepackage{graphicx}
\makeatletter
\newsavebox\pandoc@box
\newcommand*\pandocbounded[1]{% scales image to fit in text height/width
  \sbox\pandoc@box{#1}%
  \Gscale@div\@tempa{\textheight}{\dimexpr\ht\pandoc@box+\dp\pandoc@box\relax}%
  \Gscale@div\@tempb{\linewidth}{\wd\pandoc@box}%
  \ifdim\@tempb\p@<\@tempa\p@\let\@tempa\@tempb\fi% select the smaller of both
  \ifdim\@tempa\p@<\p@\scalebox{\@tempa}{\usebox\pandoc@box}%
  \else\usebox{\pandoc@box}%
  \fi%
}
% Set default figure placement to htbp
\def\fps@figure{htbp}
\makeatother





\setlength{\emergencystretch}{3em} % prevent overfull lines

\providecommand{\tightlist}{%
  \setlength{\itemsep}{0pt}\setlength{\parskip}{0pt}}



 


\KOMAoption{captions}{tableheading}
\makeatletter
\@ifpackageloaded{caption}{}{\usepackage{caption}}
\AtBeginDocument{%
\ifdefined\contentsname
  \renewcommand*\contentsname{Table of contents}
\else
  \newcommand\contentsname{Table of contents}
\fi
\ifdefined\listfigurename
  \renewcommand*\listfigurename{List of Figures}
\else
  \newcommand\listfigurename{List of Figures}
\fi
\ifdefined\listtablename
  \renewcommand*\listtablename{List of Tables}
\else
  \newcommand\listtablename{List of Tables}
\fi
\ifdefined\figurename
  \renewcommand*\figurename{Figure}
\else
  \newcommand\figurename{Figure}
\fi
\ifdefined\tablename
  \renewcommand*\tablename{Table}
\else
  \newcommand\tablename{Table}
\fi
}
\@ifpackageloaded{float}{}{\usepackage{float}}
\floatstyle{ruled}
\@ifundefined{c@chapter}{\newfloat{codelisting}{h}{lop}}{\newfloat{codelisting}{h}{lop}[chapter]}
\floatname{codelisting}{Listing}
\newcommand*\listoflistings{\listof{codelisting}{List of Listings}}
\makeatother
\makeatletter
\makeatother
\makeatletter
\@ifpackageloaded{caption}{}{\usepackage{caption}}
\@ifpackageloaded{subcaption}{}{\usepackage{subcaption}}
\makeatother
\usepackage{bookmark}
\IfFileExists{xurl.sty}{\usepackage{xurl}}{} % add URL line breaks if available
\urlstyle{same}
\hypersetup{
  pdftitle={Personality Traits vs Personality Disorders},
  colorlinks=true,
  linkcolor={blue},
  filecolor={Maroon},
  citecolor={Blue},
  urlcolor={Blue},
  pdfcreator={LaTeX via pandoc}}


\title{Personality Traits vs Personality Disorders}
\author{}
\date{}
\begin{document}
\maketitle


\subsection{Personality Traits vs Personality
Disorders}\label{personality-traits-vs-personality-disorders}

\begin{itemize}
\tightlist
\item
  This lesson explains what psychologists mean by personality traits.
\item
  This lesson shows how personality disorders are different from traits.
\item
  This lesson helps you avoid casually labeling people with disorders.
\end{itemize}

\emph{Speaker notes:}\\
State the main purpose clearly: we are drawing a careful line between
normal personality differences and personality disorders. Emphasize that
the goal is understanding, not diagnosing people in your life.

\begin{center}\rule{0.5\linewidth}{0.5pt}\end{center}

\subsection{Learning Goals}\label{learning-goals}

\begin{itemize}
\tightlist
\item
  Define personality and personality traits.
\item
  Describe what a personality disorder is.
\item
  Explain the key differences between traits and personality disorders.
\item
  Apply these ideas to clear, real-world examples.
\end{itemize}

\emph{Speaker notes:}\\
You can ask students to restate these in their own words. Say up front:
by the end, you should be able to explain this to a friend without using
technical terms.

\begin{center}\rule{0.5\linewidth}{0.5pt}\end{center}

\subsection{What Is Personality?}\label{what-is-personality}

\begin{itemize}
\tightlist
\item
  Personality is a relatively stable pattern of how you think, feel, and
  behave.
\item
  This pattern shows up across many situations and over time.
\item
  Personality helps explain why different people respond differently to
  the same event.
\end{itemize}

\emph{Speaker notes:}\\
Give a simple example: two students get the same grade; one shrugs it
off, one rewrites their whole plan. Same event, different personality
patterns. Keep it concrete.

\begin{center}\rule{0.5\linewidth}{0.5pt}\end{center}

\subsection{What Are Personality
Traits?}\label{what-are-personality-traits}

\begin{itemize}
\tightlist
\item
  Personality traits are basic dimensions of how people tend to act,
  think, and feel.
\item
  Traits are like sliders that can be higher or lower in different
  people.
\item
  A wide range of trait levels is normal in the general population.
\end{itemize}

\emph{Speaker notes:}\\
If you like, briefly name common trait ideas (such as being more
outgoing or more reserved, more organized or more spontaneous). Stress
that none of these are automatically ``good'' or ``bad.''

\begin{center}\rule{0.5\linewidth}{0.5pt}\end{center}

\subsection{Healthy Traits: Stable but
Flexible}\label{healthy-traits-stable-but-flexible}

\begin{itemize}
\tightlist
\item
  Healthy personality traits are fairly stable, but they can still flex.
\item
  People adjust how they act based on the situation and their goals.
\item
  Flexibility helps people solve problems, work with others, and
  maintain relationships.
\end{itemize}

\emph{Speaker notes:}\\
Quick think-pair-share: ``Name one trait you see in yourself most of the
time. Now, describe a situation where you act differently on purpose.''
Use their answers to highlight flexibility as a sign of health.

\begin{center}\rule{0.5\linewidth}{0.5pt}\end{center}

\subsection{When Do Traits Become a
Problem?}\label{when-do-traits-become-a-problem}

\begin{itemize}
\tightlist
\item
  A strong trait is not a disorder by itself.
\item
  A trait becomes a problem when it is extremely intense, very rigid,
  and hard to control.
\item
  The key question is whether the pattern is getting in the way of daily
  life.
\end{itemize}

\emph{Speaker notes:}\\
Say this line clearly: ``The issue is not how strong the trait is. The
issue is whether it is rigid and causing problems.'' This sets up the
transition to personality disorders.

\begin{center}\rule{0.5\linewidth}{0.5pt}\end{center}

\subsection{What Is a Personality
Disorder?}\label{what-is-a-personality-disorder}

\begin{itemize}
\tightlist
\item
  A personality disorder is a long-term pattern of thinking, feeling,
  and behaving that is inflexible.
\item
  This pattern causes serious problems in work, school, or
  relationships, or strong distress.
\item
  The pattern usually begins by adolescence or early adulthood and
  remains over time.
\end{itemize}

\emph{Speaker notes:}\\
This slide is your simplified DSM-style definition. Emphasize
``long-term,'' ``inflexible,'' and ``causes serious problems.'' You can
even underline or highlight those words when you teach.

\begin{center}\rule{0.5\linewidth}{0.5pt}\end{center}

\subsection{Three Core Features of Personality
Disorders}\label{three-core-features-of-personality-disorders}

\begin{itemize}
\tightlist
\item
  The pattern is rigid: the person acts this way across many situations.
\item
  The pattern is impairing: it causes real problems or suffering in
  life.
\item
  The pattern is pervasive and long-lasting: it has been present for
  years, not just during one stressful time.
\end{itemize}

\emph{Speaker notes:}\\
Return to these three features often. When a student uses a label like
``narcissist,'' gently ask: ``Is it rigid? Is it impairing? Is it
pervasive and long-term?''

\begin{center}\rule{0.5\linewidth}{0.5pt}\end{center}

\subsection{Traits vs Personality Disorders: Simple
Summary}\label{traits-vs-personality-disorders-simple-summary}

\begin{itemize}
\tightlist
\item
  Personality traits: patterns that describe how a person usually tends
  to be.
\item
  Personality disorders: long-term, rigid patterns that cause serious
  problems.
\item
  Traits flex with the situation; personality disorders do not flex
  much.
\item
  Traits can describe people, but only trained professionals diagnose
  disorders.
\end{itemize}

\emph{Speaker notes:}\\
Have students write or say a one-sentence version: ``Traits flex;
disorders are rigid and impairing.'' Encourage them to keep the ``only
professionals diagnose'' idea in mind.

\begin{center}\rule{0.5\linewidth}{0.5pt}\end{center}

\subsection{Example 1: Conscientiousness vs Obsessive-Compulsive
Personality
Disorder}\label{example-1-conscientiousness-vs-obsessive-compulsive-personality-disorder}

\begin{itemize}
\tightlist
\item
  Conscientiousness (trait):

  \begin{itemize}
  \tightlist
  \item
    Being organized, reliable, and careful with details.
  \item
    Can help with school, work, and responsibilities.
  \end{itemize}
\item
  Obsessive-Compulsive Personality Disorder (OCPD):

  \begin{itemize}
  \tightlist
  \item
    Perfectionism and control that slow down or block tasks.
  \item
    Difficulty delegating; relationships may suffer due to rigid rules.
  \end{itemize}
\end{itemize}

\emph{Speaker notes:}\\
Give concrete contrast:\\
Student A likes neat notes and plans ahead. Student B rewrites every
assignment repeatedly to make it perfect and often misses deadlines.
Student B's pattern is impairing and more rigid, which is what points
toward a possible disorder.

\begin{center}\rule{0.5\linewidth}{0.5pt}\end{center}

\subsection{Example 2: Extraversion vs Histrionic Personality
Disorder}\label{example-2-extraversion-vs-histrionic-personality-disorder}

\begin{itemize}
\tightlist
\item
  Extraversion (trait):

  \begin{itemize}
  \tightlist
  \item
    Enjoys being around people and social activity.
  \item
    Feels energized by talking and sharing with others.
  \end{itemize}
\item
  Histrionic Personality Disorder:

  \begin{itemize}
  \tightlist
  \item
    Constant need to be the center of attention.
  \item
    Very dramatic emotions that make relationships feel shallow or
    unstable.
  \end{itemize}
\end{itemize}

\emph{Speaker notes:}\\
Ask: ``Is liking attention a problem by itself?'' Answer: No.~The issue
is when attention-seeking is constant, extreme, and damages
relationships. Connect again to impairment and rigidity.

\begin{center}\rule{0.5\linewidth}{0.5pt}\end{center}

\subsection{Example 3: Emotional Sensitivity vs Borderline Personality
Disorder}\label{example-3-emotional-sensitivity-vs-borderline-personality-disorder}

\begin{itemize}
\tightlist
\item
  Emotional sensitivity (trait):

  \begin{itemize}
  \tightlist
  \item
    Feels emotions strongly.
  \item
    May worry or react more, but can still manage feelings most of the
    time.
  \end{itemize}
\item
  Borderline Personality Disorder:

  \begin{itemize}
  \tightlist
  \item
    Very intense and rapidly shifting emotions.
  \item
    Unstable relationships, fear of abandonment, and risk of self-harm
    or impulsive actions.
  \end{itemize}
\end{itemize}

\emph{Speaker notes:}\\
Normalize emotional sensitivity: many people are ``highly sensitive''
without having a disorder. Emphasize that Borderline Personality
Disorder involves repeated crises, serious risk, and deep impairment,
not just ``feeling a lot.''

\begin{center}\rule{0.5\linewidth}{0.5pt}\end{center}

\subsection{Why Casual Labels Are a
Problem}\label{why-casual-labels-are-a-problem}

\begin{itemize}
\tightlist
\item
  People often say ``I am so OCD'' or ``They are such a narcissist''
  about everyday behavior.
\item
  These casual labels are usually inaccurate and can add stigma to real
  conditions.
\item
  Using precise language shows respect for people who live with
  diagnosed disorders.
\end{itemize}

\emph{Speaker notes:}\\
Invite a few examples students have heard (without naming people in the
room). Use the three features (rigid, impairing, long-lasting) to show
why most casual uses do not match a true disorder.

\begin{center}\rule{0.5\linewidth}{0.5pt}\end{center}

\subsection{Check Your Understanding}\label{check-your-understanding}

\begin{itemize}
\tightlist
\item
  How would you explain the difference between a trait and a personality
  disorder in your own words?
\item
  What are the three core features that suggest a personality disorder
  instead of just a strong trait?
\item
  Why is it important not to diagnose friends, family, or yourself based
  only on a few behaviors?
\end{itemize}

\emph{Speaker notes:}\\
Use this as an exit ticket, quick write, or small-group share. Listen
for mentions of rigidity, impairment, and pervasiveness. Correct any
answers that blur those lines.

\begin{center}\rule{0.5\linewidth}{0.5pt}\end{center}

\subsection{Key Takeaways}\label{key-takeaways}

\begin{itemize}
\tightlist
\item
  Personality traits are normal differences in how people tend to think,
  feel, and behave.
\item
  Healthy traits are stable but flexible; people can adjust their
  behavior to fit the situation.
\item
  Personality disorders are long-term, rigid patterns that cause serious
  problems or distress.
\item
  Strong traits alone do not equal a disorder; the difference is
  rigidity, impairment, and how widely the pattern shows up.
\end{itemize}

\emph{Speaker notes:}\\
End by returning to the core idea: ``Traits flex. Disorders are rigid
and impairing.'' Connect this lesson to upcoming material on specific
disorders and remind them that diagnosis is a careful, professional
process.

\subsection{Cluster B Personality Disorders: Traits vs
Disorders}\label{cluster-b-personality-disorders-traits-vs-disorders}

\begin{itemize}
\tightlist
\item
  This lesson focuses on cluster B personality disorders.
\item
  We will look at how normal traits connect to these disorders.
\item
  We will practice telling strong traits apart from personality
  disorders.
\end{itemize}

\emph{Speaker notes:}\\
Cluster B includes antisocial, borderline, histrionic, and narcissistic
personality disorders. These are often described as dramatic, highly
emotional, or erratic in style. Emphasize that we are not here to
diagnose anyone in the room; we are learning how psychologists think
about traits and disorders.

\begin{center}\rule{0.5\linewidth}{0.5pt}\end{center}

\subsection{Quick Review: Traits vs Personality
Disorders}\label{quick-review-traits-vs-personality-disorders}

\begin{itemize}
\tightlist
\item
  Personality traits are normal differences in how people tend to think,
  feel, and behave.
\item
  Healthy traits are fairly stable, but they can still flex with the
  situation.
\item
  A personality disorder is a long-term, rigid pattern that causes
  serious problems in life.
\end{itemize}

\emph{Speaker notes:}\\
Brief recap from the previous lesson. If needed, ask a student to give a
one-sentence difference between a trait and a disorder. Aim to hear
words like ``rigid'' and ``impairing.''

\begin{center}\rule{0.5\linewidth}{0.5pt}\end{center}

\subsection{What Is Cluster B?}\label{what-is-cluster-b}

\begin{itemize}
\tightlist
\item
  Cluster B personality disorders share a dramatic, emotional, or
  erratic style.
\item
  People may struggle with impulse control and unstable relationships.
\item
  Cluster B includes:

  \begin{itemize}
  \tightlist
  \item
    Antisocial Personality Disorder
  \item
    Borderline Personality Disorder
  \item
    Histrionic Personality Disorder
  \item
    Narcissistic Personality Disorder
  \end{itemize}
\end{itemize}

\emph{Speaker notes:}\\
Stress that ``dramatic'' and ``emotional'' are not insults. They
describe an overall style. The key question is whether patterns are
long-term, inflexible, and impairing.

\begin{center}\rule{0.5\linewidth}{0.5pt}\end{center}

\subsection{Cluster B: Traits That Sit
Nearby}\label{cluster-b-traits-that-sit-nearby}

\begin{itemize}
\tightlist
\item
  Normal traits that sit near cluster B disorders include:

  \begin{itemize}
  \tightlist
  \item
    Boldness and risk-taking
  \item
    Emotional intensity and sensitivity
  \item
    Lively, expressive, or dramatic social style
  \item
    Strong need for approval or recognition
  \end{itemize}
\item
  These traits can be strengths when they are flexible and well managed.
\end{itemize}

\emph{Speaker notes:}\\
Have students brainstorm positive versions of each: boldness as
leadership, emotional intensity as empathy, drama as performance skills,
desire for approval as motivation to do well.

\begin{center}\rule{0.5\linewidth}{0.5pt}\end{center}

\subsection{When Nearby Traits Become
Disorders}\label{when-nearby-traits-become-disorders}

\begin{itemize}
\tightlist
\item
  Traits become a personality disorder when:

  \begin{itemize}
  \tightlist
  \item
    The pattern is rigid and shows up across many situations.
  \item
    The pattern causes serious distress or repeated life problems.
  \item
    The pattern has been present for years, not just during one crisis.
  \end{itemize}
\item
  Cluster B disorders often involve trouble with emotional control,
  impulses, and relationships.
\end{itemize}

\emph{Speaker notes:}\\
Anchor again: rigidity, impairment, and long-term presence. These three
features are your guardrails for the rest of the lecture.

\begin{center}\rule{0.5\linewidth}{0.5pt}\end{center}

\subsection{Antisocial Style: Normal Trait
Range}\label{antisocial-style-normal-trait-range}

\begin{itemize}
\tightlist
\item
  Some people are more daring and less anxious than others.
\item
  They may enjoy taking risks and breaking small rules for excitement.
\item
  In the normal range, they can still:

  \begin{itemize}
  \tightlist
  \item
    Follow basic laws and rules.
  \item
    Care about not harming others.
  \item
    Learn from consequences.
  \end{itemize}
\end{itemize}

\emph{Speaker notes:}\\
Give examples like legal extreme sports, harmless pranks, or
entrepreneurs who tolerate financial risk. These may be bold or
rule-bending, but they still respect others' rights and the law.

\begin{center}\rule{0.5\linewidth}{0.5pt}\end{center}

\subsection{Antisocial Personality
Disorder}\label{antisocial-personality-disorder}

\begin{itemize}
\tightlist
\item
  Antisocial Personality Disorder involves:

  \begin{itemize}
  \tightlist
  \item
    A pattern of ignoring the rights and safety of others.
  \item
    Repeated lying, cheating, or breaking the law.
  \item
    Little or no remorse after hurting others.
  \end{itemize}
\item
  The pattern begins early in life and continues into adulthood.
\end{itemize}

\emph{Speaker notes:}\\
Highlight the moral and legal line: repeated violation of the rights of
others, not just being rebellious. Note that many people with this
diagnosis have a history of conduct problems in childhood and ongoing
legal or work problems as adults. Emphasize impairment and harm to
others.

\begin{center}\rule{0.5\linewidth}{0.5pt}\end{center}

\subsection{Antisocial Traits vs Antisocial Personality
Disorder}\label{antisocial-traits-vs-antisocial-personality-disorder}

\begin{itemize}
\tightlist
\item
  Strong risk-taking or rule-testing is not enough for a diagnosis.
\item
  Antisocial Personality Disorder requires:

  \begin{itemize}
  \tightlist
  \item
    Long-term, pervasive disregard for others.
  \item
    Serious behavior problems, often including crime or aggression.
  \item
    A pattern of not caring about the damage done.
  \end{itemize}
\item
  Only trained professionals can diagnose this disorder.
\end{itemize}

\emph{Speaker notes:}\\
Ask: ``Can someone be rebellious or thrill-seeking without having
Antisocial Personality Disorder?'' Make sure students can answer yes and
name what is missing: the chronic, harmful disregard for others.

\begin{center}\rule{0.5\linewidth}{0.5pt}\end{center}

\subsection{Borderline Style: Normal Trait
Range}\label{borderline-style-normal-trait-range}

\begin{itemize}
\tightlist
\item
  Some people are naturally emotionally sensitive.
\item
  They may feel highs and lows more intensely than those around them.
\item
  In the normal range, they can:

  \begin{itemize}
  \tightlist
  \item
    Calm down with time or support.
  \item
    Maintain stable relationships overall.
  \item
    Keep impulses in check most of the time.
  \end{itemize}
\end{itemize}

\emph{Speaker notes:}\\
Normalize emotional sensitivity. Many students will recognize themselves
here. Emotional intensity alone is not a disorder.

\begin{center}\rule{0.5\linewidth}{0.5pt}\end{center}

\subsection{Borderline Personality
Disorder}\label{borderline-personality-disorder}

\begin{itemize}
\tightlist
\item
  Borderline Personality Disorder involves:

  \begin{itemize}
  \tightlist
  \item
    Strong and rapidly shifting emotions.
  \item
    Unstable or intense relationships and fear of abandonment.
  \item
    An unstable sense of self and frequent impulsive behaviors.
  \item
    Risk of self-harm or suicidal behavior.
  \end{itemize}
\item
  The pattern creates serious distress and relationship chaos.
\end{itemize}

\emph{Speaker notes:}\\
Be matter-of-fact but clear about the risk of self-harm and suicide.
Emphasize compassion. People with Borderline Personality Disorder are
often highly distressed and not simply ``being dramatic.''

\begin{center}\rule{0.5\linewidth}{0.5pt}\end{center}

\subsection{Emotional Sensitivity vs Borderline Personality
Disorder}\label{emotional-sensitivity-vs-borderline-personality-disorder}

\begin{itemize}
\tightlist
\item
  Emotional sensitivity means feeling things strongly, but still being
  able to cope most of the time.
\item
  Borderline Personality Disorder means:

  \begin{itemize}
  \tightlist
  \item
    Frequent emotional storms that are hard to calm.
  \item
    Repeated relationship crises and frantic efforts to avoid being
    left.
  \item
    Impulsive acts that create serious harm.
  \end{itemize}
\item
  The difference is the level of instability and impairment.
\end{itemize}

\emph{Speaker notes:}\\
Ask students: ``What signs tell you this is more than just a `big
feelings' person?'' Look for answers about repeated crises, self-harm,
and very unstable relationships.

\begin{center}\rule{0.5\linewidth}{0.5pt}\end{center}

\subsection{Histrionic Style: Normal Trait
Range}\label{histrionic-style-normal-trait-range}

\begin{itemize}
\tightlist
\item
  Some people are naturally lively, expressive, and love being the
  center of attention.
\item
  They may enjoy performing, telling stories, or dressing in bold ways.
\item
  In the normal range, they can:

  \begin{itemize}
  \tightlist
  \item
    Turn the spotlight off when needed.
  \item
    Respect social and professional boundaries.
  \item
    Maintain stable friendships and work roles.
  \end{itemize}
\end{itemize}

\emph{Speaker notes:}\\
Use entertainers as a neutral example of high expressiveness.
Attention-seeking in itself is not a disorder; context and control
matter.

\begin{center}\rule{0.5\linewidth}{0.5pt}\end{center}

\subsection{Histrionic Personality
Disorder}\label{histrionic-personality-disorder}

\begin{itemize}
\tightlist
\item
  Histrionic Personality Disorder involves:

  \begin{itemize}
  \tightlist
  \item
    Constant need to be the center of attention.
  \item
    Very dramatic or rapidly shifting emotions.
  \item
    Behavior that may be overly sexual or inappropriate for the
    situation.
  \item
    Relationships that feel shallow or only surface-deep.
  \end{itemize}
\end{itemize}

\emph{Speaker notes:}\\
Emphasize that people with this disorder may not see their behavior as a
problem. The impairment often shows up in repeated work issues and
unstable relationships.

\begin{center}\rule{0.5\linewidth}{0.5pt}\end{center}

\subsection{Expressive Traits vs Histrionic Personality
Disorder}\label{expressive-traits-vs-histrionic-personality-disorder}

\begin{itemize}
\tightlist
\item
  Being outgoing and dramatic can be a strength in many settings.
\item
  Histrionic Personality Disorder means:

  \begin{itemize}
  \tightlist
  \item
    Attention-seeking is constant and intense.
  \item
    Emotional displays are out of proportion to the situation.
  \item
    Relationships suffer because others feel used or overwhelmed.
  \end{itemize}
\item
  The problem is not ``liking attention''; it is the rigid, costly
  pattern.
\end{itemize}

\emph{Speaker notes:}\\
Have students generate two examples: one of a healthy expressive person
and one of a pattern that would fit a disorder. Help them notice the
role of context and consequences.

\begin{center}\rule{0.5\linewidth}{0.5pt}\end{center}

\subsection{Narcissistic Style: Normal Trait
Range}\label{narcissistic-style-normal-trait-range}

\begin{itemize}
\tightlist
\item
  Some people are very confident and ambitious.
\item
  They may enjoy leadership, achievement, and recognition.
\item
  In the normal range, they can:

  \begin{itemize}
  \tightlist
  \item
    Take feedback and adjust when needed.
  \item
    Recognize others' needs and contributions.
  \item
    Form real, mutual relationships.
  \end{itemize}
\end{itemize}

\emph{Speaker notes:}\\
This is a good place to point out that confidence and pride in
achievements can be healthy and motivating.

\begin{center}\rule{0.5\linewidth}{0.5pt}\end{center}

\subsection{Narcissistic Personality
Disorder}\label{narcissistic-personality-disorder}

\begin{itemize}
\tightlist
\item
  Narcissistic Personality Disorder involves:

  \begin{itemize}
  \tightlist
  \item
    A grandiose sense of self-importance.
  \item
    A strong need for admiration and praise.
  \item
    A lack of empathy for other people's feelings or needs.
  \item
    A pattern of entitlement and using others for personal gain.
  \end{itemize}
\end{itemize}

\emph{Speaker notes:}\\
Stress that this is more than ``being a little full of yourself.'' It is
a long-standing pattern across many situations, and it often leads to
work problems and relationship breakdowns.

\begin{center}\rule{0.5\linewidth}{0.5pt}\end{center}

\subsection{Confidence vs Narcissistic Personality
Disorder}\label{confidence-vs-narcissistic-personality-disorder}

\begin{itemize}
\tightlist
\item
  Healthy confidence means valuing yourself while still caring about
  others.
\item
  Narcissistic Personality Disorder means:

  \begin{itemize}
  \tightlist
  \item
    Believing you are superior and deserve special treatment.
  \item
    Becoming angry or hurt when not praised or admired.
  \item
    Ignoring or dismissing other people's needs and feelings.
  \end{itemize}
\item
  The key issues are entitlement, lack of empathy, and harm to
  relationships.
\end{itemize}

\emph{Speaker notes:}\\
Ask: ``How is a confident person different from someone with
Narcissistic Personality Disorder?'' Look for: ability to accept
criticism, empathy, and real mutual relationships.

\begin{center}\rule{0.5\linewidth}{0.5pt}\end{center}

\subsection{Common Threads Across Cluster
B}\label{common-threads-across-cluster-b}

\begin{itemize}
\tightlist
\item
  Normal traits near cluster B can include:

  \begin{itemize}
  \tightlist
  \item
    Boldness and risk-taking.
  \item
    Emotional intensity.
  \item
    Expressiveness and desire for attention.
  \item
    Ambition and confidence.
  \end{itemize}
\item
  In disorders, these traits become:

  \begin{itemize}
  \tightlist
  \item
    Rigid, extreme, and hard to control.
  \item
    Damaging to work, school, and relationships.
  \item
    Sources of distress and sometimes danger.
  \end{itemize}
\end{itemize}

\emph{Speaker notes:}\\
Summarize: Cluster B is not about having big feelings or big
personalities. It is about patterns that are intense, rigid, and
impairing across many parts of life.

\begin{center}\rule{0.5\linewidth}{0.5pt}\end{center}

\subsection{Why You Should Not Diagnose People Around
You}\label{why-you-should-not-diagnose-people-around-you}

\begin{itemize}
\tightlist
\item
  Many people show bits and pieces of these traits at times.
\item
  Only trained professionals can diagnose personality disorders.
\item
  Mislabeling others as ``narcissistic,'' ``borderline,'' or
  ``antisocial'' can be:

  \begin{itemize}
  \tightlist
  \item
    Inaccurate
  \item
    Stigmatizing
  \item
    Harmful to real help-seeking
  \end{itemize}
\end{itemize}

\emph{Speaker notes:}\\
Reinforce humility. Encourage students to use this knowledge to
understand patterns, not to attack or pathologize people in their lives.
If time allows, discuss why people might casually use these labels and
what the risks are.

\begin{center}\rule{0.5\linewidth}{0.5pt}\end{center}

\subsection{Check Your Understanding}\label{check-your-understanding-1}

\begin{itemize}
\tightlist
\item
  How can you tell the difference between a normal trait and a
  personality disorder?
\item
  What makes antisocial, borderline, histrionic, and narcissistic
  patterns part of the same cluster?
\item
  Why is flexibility such an important idea in deciding whether a trait
  has become a disorder?
\end{itemize}

\emph{Speaker notes:}\\
Use these as quick-write prompts or a short quiz. Listen for
understanding of the trait--disorder boundary, the dramatic and
emotional style of cluster B, and the role of flexibility versus
rigidity.

\begin{center}\rule{0.5\linewidth}{0.5pt}\end{center}

\subsection{Key Takeaways}\label{key-takeaways-1}

\begin{itemize}
\tightlist
\item
  Cluster B personality disorders share a dramatic, emotional, or
  erratic style.
\item
  Each disorder has nearby traits that can be normal and even helpful in
  the right context.
\item
  The shift from trait to disorder happens when patterns are rigid,
  long-term, and seriously impairing.
\item
  Understanding these differences can reduce stigma and improve how we
  talk about mental health.
\end{itemize}

\emph{Speaker notes:}\\
End by returning to the core teaching idea: traits can be strong and
still healthy; disorders are strong, rigid, and costly across many parts
of life. Connect forward to upcoming units on other psychological
disorders and treatment.




\end{document}
